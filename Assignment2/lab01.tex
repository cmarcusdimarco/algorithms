%%%%%%%%%%%%%%%%%%%%%%%%%%%%%%%%%%%%%%%%%
%
% CMPT 435
% Lab One
%
%%%%%%%%%%%%%%%%%%%%%%%%%%%%%%%%%%%%%%%%%

%%%%%%%%%%%%%%%%%%%%%%%%%%%%%%%%%%%%%%%%%
% Short Sectioned Assignment
% LaTeX Template
% Version 1.0 (5/5/12)
%
% This template has been downloaded from: http://www.LaTeXTemplates.com
% Original author: % Frits Wenneker (http://www.howtotex.com)
% License: CC BY-NC-SA 3.0 (http://creativecommons.org/licenses/by-nc-sa/3.0/)
% Modified by Alan G. Labouseur  - alan@labouseur.com
%
%%%%%%%%%%%%%%%%%%%%%%%%%%%%%%%%%%%%%%%%%

%----------------------------------------------------------------------------------------
%	PACKAGES AND OTHER DOCUMENT CONFIGURATIONS
%----------------------------------------------------------------------------------------

\documentclass[letterpaper, 10pt]{article} 

\usepackage[english]{babel} % English language/hyphenation
\usepackage{graphicx}
\usepackage[lined,linesnumbered,commentsnumbered]{algorithm2e}
\usepackage{listings}
\usepackage{fancyhdr} % Custom headers and footers
\pagestyle{fancyplain} % Makes all pages in the document conform to the custom headers and footers
\usepackage{lastpage}
\usepackage{url}
\usepackage{xcolor}
\usepackage{multirow}

\fancyhead{} % No page header - if you want one, create it in the same way as the footers below
\fancyfoot[L]{} % Empty left footer
\fancyfoot[C]{page \thepage\ of \pageref{LastPage}} % Page numbering for center footer
\fancyfoot[R]{}

\definecolor{commentgreen}{rgb}{0, 0.6, 0}
\definecolor{keywordmagenta}{rgb}{.84, 0.09, 0.41}

\lstdefinestyle{mystyle}{
    commentstyle=\color{commentgreen},
    keywordstyle=\color{keywordmagenta},
    numberstyle=\color{gray},
    numbers=left,
    literate={\ \ }{{\ }}1
}

\lstset{style=mystyle}

\renewcommand{\headrulewidth}{0pt} % Remove header underlines
\renewcommand{\footrulewidth}{0pt} % Remove footer underlines
\setlength{\headheight}{13.6pt} % Customize the height of the header

%----------------------------------------------------------------------------------------
%	TITLE SECTION
%----------------------------------------------------------------------------------------

\newcommand{\horrule}[1]{\rule{\linewidth}{#1}} % Create horizontal rule command with 1 argument of height

\title{	
   \normalfont \normalsize 
   \textsc{CMPT 435 - Fall 2021 - Dr. Labouseur} \\[10pt] % Header stuff.
   \horrule{0.5pt} \\[0.25cm] 	% Top horizontal rule
   \huge Assignment Two -- Magic Items Sorter \\     	    % Assignment title
   \horrule{0.5pt} \\[0.25cm] 	% Bottom horizontal rule
}

\author{C. Marcus DiMarco \\ \normalsize C.DiMarco1@Marist.edu}

\date{\normalsize\ October 8, 2021} 	% Today's date.

\begin{document}

\maketitle % Print the title

%----------------------------------------------------------------------------------------
%   CONTENT SECTION
%----------------------------------------------------------------------------------------

% - -- -  - -- -  - -- -  -

\section{Objectives}

\begin{itemize}
    \item \hspace{0.5em}Create a program that sorts all Strings within a given text file using selection sort, insertion sort, merge sort and Quicksort.
    \item \hspace{0.5em}Print the number of comparisons for each sort.
\end{itemize}

\vspace{1.0em}

\section{Conditions}

\begin{itemize}
    \item \hspace{0.5em}The relation on the set of Strings must be 'less than', such that for any element a which appears before any other element b, a is less than b. For Strings, 'less than' is defined as preceding alphabetically.
    \item \hspace{0.5em}The array must be shuffled before running the sorts, and the sorts must act on the same shuffle of the array.
    \item \hspace{0.5em}The program must only use custom algorithms for the sorts. Any of these sorts which already exist in the language may not be used.
\end{itemize}

\vspace{1.0em}

\section{Overview}

\hspace{1.0em}In order to enhance readability, each sorting algorithm is a method contained in the class MarcusSort.

\subsection{selectionSort()}

\hspace{1.0em}The most straightforward sorting algorithm in this program is selection sort. Selection sort starts at the first index, finds the minimum value in the array and places it at the current index, repeating until the array is sorted.

\begin{lstlisting}[language=Java, firstnumber=16]
    // Selection sort will search for minimum unsorted value and
    // place it at first unsorted index.
    public void selectionSort(String[] strings) {

        // Start with counter at 0
        this.resetCounter();

        // First unsorted index through each pass is i
        for (int i = 0; i < strings.length; i++) {
            counter++;          // Comparison in for loop
            int minIndex = i;
            // Find index of smallest value in remainder of array
            for (int j = i + 1; j < strings.length; j++) {
                counter += 2;   // Comparisons in for loop and if statement
                if (strings[minIndex].compareToIgnoreCase(strings[j]) > 0) {
                    minIndex = j;
                }
            }
            counter++;          // Comparison for inner loop termination

            // Swap indices minIndex and i
            String temp = strings[i];
            strings[i] = strings[minIndex];
            strings[minIndex] = temp;
        }
        counter++;              // Comparison for outer loop termination

        // Print completion message with number of comparisons
        this.printCompletionMessage();
    }
\end{lstlisting}

\subsection{insertionSort()}

\hspace{1.0em}Improving on selection sort is insertion sort. Insertion sort relies on taking a key value and placing it in sorted order among all of the previous elements. We start at index 1 and compare it to the value at index 0, sorting those two as needed. Note that after each iteration, indices 0-i are sorted.

\vspace{3.0em}

\begin{lstlisting}[language=Java, firstnumber=45]
    // Insertion sort will sort the array from index 0 through i,
    // 'inserting' each value into its correct position.
    public void insertionSort(String[] strings) {

        // Start with counter at 0
        this.resetCounter();

        // Start comparisons at index 1
        for (int i = 1; i < strings.length; i++) {
            counter++;      // Comparison in outer loop
            String keyString = strings[i];
            for (int j = i - 1; j >= 0; j--) {
                counter += 2;   // Inner loop and if statement
                // Move all items larger than key forward 1 index
                if (keyString.compareToIgnoreCase(strings[j]) < 0) {
                    strings[j + 1] = strings[j];
                    counter++;  // Second comparison in if statement
                    // If index 0 reached, assign it currentString
                    if (j == 0) {
                        strings[j] = keyString;
                    }
                } else {
                    // Insert key at index ahead of first smaller element
                    strings[j + 1] = keyString;
                    break;
                }
            }
            counter++;      // Comparison for inner loop termination
        }
        counter++;          // Comparison for outer loop termination

        // Print completion message with number of comparisons
        this.printCompletionMessage();
    }
\end{lstlisting}

\subsection{mergeSort()}

\hspace{1.0em}In stark contrast to the other sorting algorithms seen so far, merge sort drastically reduces complexity by implementing a "divide and conquer" tactic, which relies on a simple principle: arrays of size 1 are sorted. Therefore, if we divide the target array into two subarrays, and continue to divide those until we reach arrays of size 1, we can recombine the subarrays in sorted order and greatly reduce the amount of work needed. It uses two operations, sortThenMerge() and merge().

\vspace{2.0em}

\begin{lstlisting}[language=Java, firstnumber=80]
    /**
     * mergeSort() is an abstraction for cleaner user interaction. It will reset
     * the counter, call the recursive merge function, and then print the
     * completion message with the number of comparisons.
     * @param strings
     */
    public void mergeSort(String[] strings) {

        // Start with counter at 0
        this.resetCounter();

        // Nest recursive function inside for readability and proper
        // counter/print calls
        this.sortThenMerge(strings, 0, strings.length - 1);

        // Print completion message with number of comparisons
        this.printCompletionMessage();
    }

    // Merge sort will divide the array recursively until subarrays are
    // of size 1, then merge the subarrays together in sorted order.
    private void sortThenMerge(String[] strings, int leftIndex, int rightIndex) {
        
        counter++;      // Increment the if comparison
        if (leftIndex < rightIndex) {
            int midpoint = (leftIndex + rightIndex) / 2;
            sortThenMerge(strings, leftIndex, midpoint);
            sortThenMerge(strings, midpoint + 1, rightIndex);
            merge(strings, leftIndex, midpoint, rightIndex);
        }
    }

    // Merge will take an array, create two sorted subarrays, and
    // sort the elements in place.
    private void merge(String[] strings, int leftIndex, int midpoint,
                       int rightIndex) {

        // Create two (sorted, due to recursion) subarrays
        String[] stringsL = new String[midpoint - leftIndex + 1];
        String[] stringsR = new String[rightIndex - midpoint];
        
        // Populate the subarrays, incrementing for each comparison
        for (int i = 0; i < stringsL.length; i++) {
            counter++;
            stringsL[i] = strings[leftIndex + i];
        }
        counter++;      // Increment the loop exit comparison

        for (int j = 0; j < stringsR.length; j++) {
            counter++;
            stringsR[j] = strings[midpoint + j + 1];
        }
        counter++;      // Increment the loop exit comparison

        int i = 0;
        int j = 0;
        int k = leftIndex;
        // For the selected range, sort until one subarray is exhausted
        while (i < stringsL.length && j < stringsR.length) {
            counter += 2;   // Increment for loop and if statements
            if (stringsL[i].compareToIgnoreCase(stringsR[j]) < 0) {
                strings[k++] = stringsL[i++];
            } else {
                strings[k++] = stringsR[j++];
            }
        }
        // Append the remaining (sorted) subarray
        if (i == stringsL.length) {
            for ( ; k <= rightIndex; k++) {
                counter++;
                strings[k] = stringsR[j++];
            }
            counter += 2;   // Increment for loop exit and if
        } else {
            for ( ; k <= rightIndex; k++) {
                counter++;
                strings[k] = stringsL[i++];
            }
            counter += 2;   // Increment for loop exit and if
        }
    }
\end{lstlisting}

\vspace{1.0em}

\subsection{quickSort()}

\hspace{1.0em}Building off of the "divide and conquer" principle in merge sort, Quicksort looks to make things even more efficient by beginning the conquering during the dividing instead of waiting until after. Quicksort selects a pivot value and divides the elements in the array around it, with all smaller elements to its left and all larger elements to its right. This means that after each call, any pivot value is in its correct index.

\vspace{2.0em}

\begin{lstlisting}[language=Java, firstnumber=162]
    /**
     * quickSort() is an abstraction for cleaner user interaction. It will reset
     * the counter, call the recursive quicksort function, and print the
     * completion message with the number of comparisons.
     * @param strings
     */
    public void quickSort(String[] strings) {

        // Set counter to 0
        this.resetCounter();

        // Nest recursive function inside for readability and proper
        // counter/print calls
        this.quickSort(strings, 0, strings.length - 1);

        // Print completion message with number of comparisons
        this.printCompletionMessage();
    }

    // Quicksort will take an array and recursively divide it around
    // a pivot value, sorting the elements around the pivots
    private void quickSort(String[] strings, int leftIndex, int rightIndex) {

        counter++;      // Increment for if comparison
        if (leftIndex < rightIndex) {
            // Quicksort the elements to either side of the partition
            int partition = partition(strings, leftIndex, rightIndex);
            quickSort(strings, leftIndex, partition - 1);
            quickSort(strings, partition + 1, rightIndex);
        }
    }

    // Partitions an array around a pivot value and places all values
    // smaller than the pivot to its left, then places pivot at its sorted index
    private int partition(String[] strings, int leftIndex, int rightIndex) {

        String pivotString = strings[rightIndex];

        int sortedIndex = leftIndex;

        for (int i = leftIndex; i <= rightIndex; i++) {
            counter += 2;       // Increment for loop and if statement
            if (strings[i].compareToIgnoreCase(pivotString) < 0) {
                // If element at i is smaller than pivot, place in the
                // left half of the array
                String tempString = strings[sortedIndex];
                strings[sortedIndex++] = strings[i];
                strings[i] = tempString;
            }
        }
        counter++;      // Increment for loop exit comparison

        // Swap pivot string and value at the pivot's sorted index
        strings[rightIndex] = strings[sortedIndex];
        strings[sortedIndex] = pivotString;
        return sortedIndex;
    }
\end{lstlisting}

\subsection{Other helper functions}

\hspace{1.0em}Below are the remaining functions present in MarcusSort().

\vspace{2.0em}

\begin{lstlisting}[language=Java, firstnumber=220]
    // Shuffle routine based on the Knuth or Fisher-Yates, but not
    // Rosanna, shuffle
    public void notRosannaShuffle(String[] strings) {

        for (int i = strings.length - 1; i >= 0; i--) {
            // Get random index in the remaining length
            int randomIndex = (int) Math.floor(Math.random() * i);
            if (randomIndex == i) {
                continue;
            } else {
                // If the random index is not i, swap their values
                String tempString = strings[i];
                strings[i] = strings[randomIndex];
                strings[randomIndex] = tempString;
            }
        }
    }

    // Getter for counter
    public int getCounter() {
        return counter;
    }

    // Controlling setter functionality to only reset to 0
    public void resetCounter() {
        counter = 0; 
    }

    // Message to print upon completing sort which includes counter
    public void printCompletionMessage() {
        System.out.println("Sort complete! Number of comparisons: "
                           + counter);
    }
\end{lstlisting}

\subsection{Assignment2}

\hspace{1.0em}With the methods we have just defined, we are ready to create our sorter. We'll need some imported libraries: namely, java.io.File to import a file, java.io.FileNotFoundException to account for errors finding the input file, and java.util.Scanner to read the file.

\vspace{2.0em}

\begin{lstlisting}[language=Java]
/**
 * A program which reads a constant number of Strings
 * from a file and sorts them using various methods,
 * comparing the efficiency of the sorts by
 * printing the number of comparisons for each.
 */
import java.io.File;
import java.io.FileNotFoundException;
import java.util.Scanner;

public class Assignment2 {
    public static void main(String[] args) {
        final int NUM_OF_ITEMS = 666;                   // Length of file as constant
        String[] magicItems = new String[NUM_OF_ITEMS]; // Array of file strings
        MarcusSort sorter = new MarcusSort();           // Instance of MarcusSort

        // Try/catch block for file import and reading
        try {
            File file = new File("magicitems.txt");
            Scanner read = new Scanner(file);
            for (int i = 0; i < NUM_OF_ITEMS; i++) {
                magicItems[i] = read.nextLine();
            }
            read.close();
        } catch (FileNotFoundException e) {
            System.out.println("Whoops! Couldn't find magicitems.txt");
            e.printStackTrace();
        }

        // Shuffle array before beginning sorts
        sorter.notRosannaShuffle(magicItems);

        // Create second array to store the shuffle
        String[] magicItemsShuffled = new String[NUM_OF_ITEMS];
        System.arraycopy(magicItems, 0, magicItemsShuffled, 0, NUM_OF_ITEMS);

        // Sort using selection sort, print comparisons, reset shuffle
        sorter.selectionSort(magicItems);
        System.arraycopy(magicItemsShuffled, 0, magicItems, 0, NUM_OF_ITEMS);

        // Sort using insertion sort, print comparisons, reset shuffle
        sorter.insertionSort(magicItems);
        System.arraycopy(magicItemsShuffled, 0, magicItems, 0, NUM_OF_ITEMS);

        // Sort using merge sort, print comparisons, reset shuffle
        sorter.mergeSort(magicItems);
        System.arraycopy(magicItemsShuffled, 0, magicItems, 0, NUM_OF_ITEMS);

        // Sort using quicksort, print comparisons
        sorter.quickSort(magicItems);
    }
}
\end{lstlisting}

\section{Results}

\hspace{1.0em}Due to the pseudorandom nature of notRosannaShuffle(), exact results will vary from execution to execution. The below table shows the results obtained during 5 trials, along with the average number of comparisons and the expected order of growth for each sort.

\begin{center}
\begin{tabular}{|c||c|c|c|c|}
    \hline
    \multicolumn{5}{|c|}{Sort Performance (in number of comparisons)} \\
    \hline
    & selectionSort() & insertionSort() & mergeSort() & quickSort() \\
    \hline\hline
    Trial 1 & 444,223 & 334,807 & 22,004 & 15,454 \\
    Trial 2 & 444,223 & 344,659 & 22,011 & 15,834 \\
    Trial 3 & 444,223 & 338,621 & 22,025 & 17,330 \\
    Trial 4 & 444,223 & 336,871 & 22,051 & 15,246 \\
    Trial 5 & 444,223 & 337,681 & 22,033 & 14,963 \\
    \hline
    Average & 444,223 & 338,527.8 & 22,024.8 & 15,765.4 \\
    \hline
    $O(g(n))$ & $O(n^2)$ & $O(n^2)$ & $O(n log(n))$ & $O(n log(n))$ \\
    \hline
\end{tabular}
\end{center}

\hspace{1.0em}The first two algorithms belong to $O(n^2)$ due to their nested loops; ignoring constant factors and lower order terms, they iterate over the array during each iteration of the sort, and the sort iterates for each term in the array. The second pair of algorithms expect to run in $O(n log(n))$ time due to their trademark "linearithmic" approach, "divide and conquer". The division is logarithmic, and the conquering/merging is linear.
\end{document}
